\documentclass[letterpaper, onecolumn, 11pt]{report}
\usepackage[utf8]{inputenc}
\usepackage[spanish]{babel}
\usepackage{amsmath}
\usepackage{graphicx}
\usepackage{physics}
\usepackage{wrapfig}
\usepackage{hyperref}
\usepackage{marginnote}
\usepackage{caption}
\usepackage{amsfonts}
%\usepackage{draculatheme} %Agregar draculatheme.sty  al directorio del proyecto LaTeX
\spanishdecimal{.}
\usepackage{hyperref}
\usepackage{marginnote}
\hypersetup{colorlinks=true, linkcolor=red}

\renewcommand*{\marginnotevadjust}{-0.1cm}
\renewcommand*{\marginfont}{\footnotesize}
\usepackage[right=4.5cm,left=2cm,top=3cm,bottom=3.0cm]{geometry}
\begin{document}
\sffamily
%\title{\Huge\textbf{{Título}}}
%\author{Diez B. Borja}
%\maketitle
%\input{cap/tensor-algebra.tex}
%\tableofcontents

\begin{equation}
	g = 1+z+\frac{z^2}{2}
\end{equation}
Construya una gráfica que exhiba el borde mencionado ($|g|=1$) mediante la evaluación de la magnitud de $g$ en el rango $x=[-3,3]$, y $y=[-3,3]$. Repita el procedimiento anterior para el método de 4to orden de RK
\begin{equation}
	g = 1+\Delta t\lambda+\frac{1}{2}(\Delta t\lambda)^2+\frac{1}{6}(\Delta t\lambda)^3+\frac{1}{24}(\Delta t\lambda)^4
\end{equation}




\end{document}

