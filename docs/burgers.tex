\documentclass[letterpaper, onecolumn, 11pt]{report}
\usepackage[utf8]{inputenc}
\usepackage[spanish]{babel}
\usepackage{amsmath}
\usepackage{graphicx}
\usepackage{physics}
\usepackage{wrapfig}
\usepackage{hyperref}
\usepackage{marginnote}
\usepackage{caption}
\usepackage{amsfonts}
%\usepackage{draculatheme} %Agregar draculatheme.sty  al directorio del proyecto LaTeX
\spanishdecimal{.}
\usepackage{hyperref}
\usepackage{marginnote}
\hypersetup{colorlinks=true, linkcolor=red}

\renewcommand*{\marginnotevadjust}{-0.1cm}
\renewcommand*{\marginfont}{\footnotesize}
\usepackage[right=4.5cm,left=2cm,top=3cm,bottom=3.0cm]{geometry}
\begin{document}
\sffamily
%\title{\Huge\textbf{{Título}}}
%\author{Diez B. Borja}
%\maketitle
%\input{cap/tensor-algebra.tex}
%\tableofcontents

\section*{La ecuación de Burgers}
La ecuación de Burgers para $u=u(x,t)$ es una ecuación de advección no-lineal en la cual la velocidad de propagación coincide con la variable $u$
\begin{equation}
	\pdv{u(x,t)}{t} + u\pdv{u(x,t)}{x} = D\pdv[2]{u(x,t)}{x}
\end{equation}
con $u(x,0)=u_0(x)$ y donde supondremos condiciones de borde periódicas, es decir $u(x,t)=u(x+Lt)$, con $L$ la longitud del dominio espacial.

Notar que si deseamos transformar la parte espacial al espacio de Fourier, primero debemos llevar la ecuación anterior a la forma
\begin{align}
	\pdv{u}{t} &= -\frac{1}2\pdv{u^2}{x} + D \pdv[2]{u}{x}\\
						 &=F(u)
\end{align}

\end{document}

